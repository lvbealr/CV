\documentclass[11pt]{report}
\usepackage[english, russian]{babel}
\usepackage[margin=0.59in]{geometry}
\usepackage{graphicx}
\usepackage{tikz}
\usetikzlibrary{calc}
\usepackage{hyperref}

\definecolor{gray}{RGB}{112, 128, 144}

\begin{document}
\begin{titlepage}

\begin{large}

\begin{tikzpicture}[remember picture,overlay]
    \draw[line width=4pt, color=gray]
        ($(current page.north west) + (0.30in,-0.30in)$) rectangle
        ($(current page.south east) + (-0.30in,0.30in)$);
\end{tikzpicture}

\raggedright \textbf{Максим Зубаха} \\ Московский физико-технический институт \\ Системное программирование и прикладная математика \\ Долгопрудный, Россия \\ \href{zubakha.ma@phystech.edu}{\textit{zubakha.ma@phystech.edu}} \\ +7 (927) 364-44-25 \\

\vspace{0.7em}

\raggedright Уважаемая команда Baikal Electronics, \\

\vspace{0.7em}

Как студент первого курса МФТИ по направлению «Системное программирование и прикладная математика», я стремлюсь внести вклад в разработку технологий, где глубокое понимание аппаратуры сочетается с оптимизацией системного ПО. Ваши проекты в области Linux-драйверов, runtime для AI-процессоров и эмуляции платформ (например, QEMU) особенно близки мне, так как они требуют работы на стыке алгоритмических решений и низкоуровневой инженерии. Мой интерес к Linux Kernel, архитектурным оптимизациям и системному программированию идеально соответствует задачам вашей компании.

\vspace{0.7em}

Из описания ваших проектов я узнал о разработке драйверов для GPGPU, оптимизации алгоритмов планирования потоков и интеграции runtime-библиотек для AI-процессоров. Эти задачи требуют не только знания специфики аппаратуры, но и умения проектировать эффективные абстракции на уровне ядра ОС. Например, при разработке рендерера множества Мандельброта на C с использованием GLSL-шейдеров я исследовал, как распараллеливание вычислений на GPU и оптимизация памяти влияют на производительность. 

\vspace{0.7em}

Сейчас я владею языком C, изучаю C++, знаком с архитектурой x86-64 (кэши, конвейеризация, виртуальная память, прерывания) и готов углубленно изучать RISC-V/ARM. Мой опыт включает:

\begin{itemize}
    \item \href{github.com/lvbealr/MandelbrotSet}{Рендеринг множества Мандельброта} с использованием GLSL-shaders и технологии многопоточности: я убедился, насколько критичны оптимизации для ускорения вычислений на GPU, какого прироста можно добиться, распараллеливая вычисления (Intel Intrinsics, multithreading)

    \item Создание \href{github.com/lvbealr/SPU}{Soft Processing Unit} (SPU): собственный проект эмулятора процессора, включающий реализацию базовых инструкций и управление памятью. Этот опыт дал мне понимание этапов проектирования вычислительных систем, что может быть полезно в ваших задачах по эмуляции платформ (QEMU).

    \item Начальную разработку \href{github.com/lvbealr/Language}{компилятора эзотерического языка}: от проектирования грамматики до генерации промежуточного кода (в стадии разработки). Работа над этим проектом углубила мои знания в области компиляторных технологий, включая анализ AST и оптимизации IR, что напрямую связано с вашими задачами по разработке LLVM-пассов и CUDA-компиляторов.

    \item Оптимизацию \href{github.com/lvbealr/HashTable}{хэш-таблицы с использованием perf и Intel Intrinsics}: анализ производительности через профилирование (perf, cachegrind) и эксперименты с разными стратегиями хэширования. Этот проект научил меня находить баланс между алгоритмической сложностью и аппаратными ограничениями — навык, критичный для ваших задач, таких как оптимизация матричных операций.

\end{itemize}

\newpage

\begin{tikzpicture}[remember picture,overlay]
    \draw[line width=4pt, color=gray]
        ($(current page.north west) + (0.30in,-0.30in)$) rectangle
        ($(current page.south east) + (-0.30in,0.30in)$);
\end{tikzpicture}

Меня особенно привлекают ваши задачи, связанные с Linux Kernel — например, поддержка драйверов GPGPU или оптимизация алгоритмов планирования потоков. Понимаю, что для таких проектов критично умение балансировать между требованиями к latency, throughput и деталями аппаратной реализации. 

\vspace{0.7em}

Baikal Electronics для меня — это шанс работать над проектами, где каждый результат имеет значение для развития. Ваша работа над процессорами и системным ПО требует сочетания теоретической строгости и практической изобретательности. Я готов активно изучать новые инструменты — от QEMU для эмуляции до нюансов RISC-V-ориентированного кода в ядре Linux.

\vspace{0.7em}

Буду рад присоединится к вашей команде!

\vspace{0.7em}

\raggedright С уважением,\\
\textbf{Максим Зубаха}

\vspace{1.0em}

\end{large}

\end{titlepage}
\end{document}